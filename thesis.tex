\documentclass[12pt, a4paper]{report}
\usepackage{polyglossia}
\setdefaultlanguage{magyar}
\usepackage{mathtools}
\usepackage{amsthm}
\usepackage{fontspec}
\usepackage{unicode-math}
\setmainfont{GentiumPlus-R.ttf}[BoldFont=GenBasB.ttf,ItalicFont=GentiumPlus-I.ttf,BoldItalicFont=GenBasBI.ttf]
\setmathfont{TeX Gyre Pagella Math}
\defaultfontfeatures{Ligatures=TeX}
\usepackage[paper=a4paper,left=3.5cm,right=2.5cm,top=2.5cm,bottom=2.5cm]{geometry}
\usepackage{graphicx}
\usepackage{setspace}
\usepackage{indentfirst}
\usepackage[backend=biber,style=numeric,sortlocale=hu,block=space,isbn=false,doi=false,url=false]{biblatex}
\addbibresource{bibliography.bib}
\usepackage[bookmarks,unicode,hyperfootnotes=false,pdfborder={0 0 0 0}]{hyperref}
\urlstyle{same}
\usepackage{titlesec}

\hypersetup{pdfauthor={Vadas Norbert},pdftitle={Élszínezések}}

\pagestyle{plain}
\onehalfspacing
\frenchspacing

\swapnumbers
\newtheorem{tét}{Tétel}[section]
\newtheorem{lem}[tét]{Lemma}
\newtheorem{áll}[tét]{Állítás}
\newtheorem{sej}[tét]{Sejtés}
\newtheorem*{köv}{Következmény}
\theoremstyle{remark}
\newtheorem*{megj}{Megjegyzés}
\theoremstyle{definition}
\newtheorem{defi}{Definíció}[section]
\newtheorem{pl}{Példa}[section]

\DefineBibliographyStrings{english}{%
  and              = {és},
}

\begin{document}
\titleformat{\chapter}[hang]
    {\normalfont\LARGE\bfseries}{\thechapter.}{1em}{}

\begin{titlepage}
\begin{center}
{\huge \textsc{Eötvös Loránd Tudományegyetem \\ Természettudományi Kar \\}}
\hrulefill \\[2.5cm]
{\huge Vadas Norbert} \\[0.7cm]
{\Huge \textsc{Élszínezések}} \\[0.7cm]
alkalmazott matematikus MSc szakdolgozat \\[0.1cm]
operációkutatás szakirány \\[2.2cm]
{\large Témavezetõ:} \\[0.4cm]
{\Large Bérczi Kristóf} \\[0.3cm] 
{\Large Operációkutatási Tanszék}
\vfill
\includegraphics[width=0.3\textwidth]{./images/elte_cimer_ff} \\[0.5cm]
{\large Budapest, 2015}
\end{center}
\end{titlepage}

\pagenumbering{roman}
\tableofcontents

\chapter{Bevezetés} 
\pagenumbering{arabic}

\section{Fogalmak és jelölések}
A továbbiakban, hacsak nincs másképp jelezve, minden gráf egyszerű, véges és irányítatlan. Egy $G = (V, E)$ gráfon értelmezett $w: E \rightarrow [k] = \lbrace 1, \ldots, k \rbrace$ függvényt \textbf{$k$-élsúlyozás}nak nevezünk. Amennyiben a csúcsokhoz is rendelünk súlyokat, azaz $w: V \cup E \rightarrow [k]$, akkor \textbf{$k$-teljes-súlyozás}ról beszélünk. Egy csúcs \textbf{érték}én vagy \textbf{szín}én a rá illeszkedő élek súlyainak, és amennyiben van, a saját súlyának összegét értjük. Az első eset egy másik elnevezése még a \textbf{súlyozott fokszám}. Azt mondjuk, hogy egy súlyozás \textbf{csúcs-megkülönböztető}, ha bármely két csúcsnak különböző az értéke. Abban az esetben, ha ezt csak szomszédos csúcspárokra követeljük meg, akkor a csúcsok értékei egy színezését adják a gráfnak. Az ilyen súlyozást \textbf{csúcs-színező}nek hívjuk. Adott $G$ gráfra a legkisebb olyan $k$ számot, melyre létezik $G$-nek csúcs-színező $k$-élsúlyozása $\chi_e(G)$-vel jelöljük. Végezetül egy gráfra azt mondjuk, hogy \textbf{rendes}, ha egyetlen komponense sem izomorf $K_2$-vel.

\section{Az 1, 2, 3 - sejtés}
Az 1, 2, 3 - sejtés vizsgálatát a gráfok irregularitásának vizsgálata motiválta. Egy gráf éleinek súlyozását irregulárisnak nevezzük, ha bármely két csúcsra a rájuk illeszkedő éleken vett összeg különböző. Egy gráf irregularitásának erősségén azt a legkisebb $k$ számot értjük, amelyre létezik irreguláris súlyozás az $\lbrace 1, \ldots, k \rbrace$ halmazból vett súlyokkal. Ennek a feladatnak egy természetesen adódó egyszerűsítése, ha csak szomszédos csúcsokra követeljük meg azt, hogy különböző legyen az értékük. 

A sejtést először \citeauthor{Karonski2004} \cite{Karonski2004} fogalmazta meg 2002-ben, és a következőképpen hangzik: 

\begin{sej}[Az 1, 2, 3 - sejtés]
Minden rendes gráf élei megcímkézhetőek az 1, 2, 3 számokkal oly módon, hogy tetszőleges két szomszédos csúcsra a rájuk illeszkedő éleken lévő számok összege különböző legyen.
\end{sej}

A sejtést megfogalmazása óta sokat vizsgálták. Az eddigi legjobb korlátot \citeauthor{Kalkowski2010} \cite{Kalkowski2010} bizonyította be 2010-ben, mely szerint a helyes színezéshez 5 élsúly elegendő. Könnyen látható, hogy léteznek olyan rendes gráfok, amelyekre nem elég 2 élsúly. Azonban egy aszimptotikus eredmény szerint egy $G(p, n)$ véletlen gráf majdnem biztosan megszínezhető csak az 1, 2 élsúlyok segítségével \cite{AddarioBerry2008}. Bizonyos gráfosztályokra már sikerült igazolni a sejtést. Eszerint 3-színezhető \cite{Karonski2004}, illetve teljes gráfok \cite{Alaeiyan2012} esetén $\chi_e(G) = 3$. Az előbbi eredmény nyomán feltehető az a kérdés, hogy mely páros gráfok esetében elegendő csak az 1, 2 súlyok közül választani. \citeauthor{Lu2011} \cite{Lu2011} cikke szerint a 3-összefüggő, valamint bizonyos fokszám-megkötéseknek eleget tevő páros gráfok ilyenek.

A csúcs-színező élsúlyozásoknak számos változatát vizsgálták már az elmúlt évtizedben. Az irányított esetben egy digráf éleit súlyozzuk, a csúcsok értékét pedig csak a kifelé vezető éleken vett összeg határozza meg. Ez a probléma lényegesen egyszerűbb, mint az irányítatlan változat, ugyanis itt könnyedén belátható az 1, 2, 3 - sejtéssel analóg állítás \cite{Baudon2014}.

\begin{áll}
Minden $D$ digráfra $\chi_e(D) = 3$.
\end{áll}

Más változatokban az élsúlyok összege helyett azok szorzata, halmaza, multihalmaza vagy sorozata határozza meg a csúcsok színeit. Emellett élsúlyozás helyett tekinthetünk csúcs-, illetve teljes-súlyozást is. Érdekes kérdés az is, hogy mit mondhatunk abban az esetben, ha a súlyokat nem az $\lbrace 1, \ldots, k \rbrace$ halmazból, hanem tetszőleges $k$-elemű listából választhatjuk ki. A különféle változatok eddigi eredményeiről \citeauthor{Seamone2012} \cite{Seamone2012} cikkében olvashatunk bővebben.

Természetesen adódik az a kérdés is, hogy vajon NP-nehéz-e annak eldöntése, hogy egy gráf színezéséhez 2 élsúly elegendő. Irányított gráfokra a válasz igen, egyéb esetben ez egy nyitott probléma.

\chapter{A fontosabb eredmények}

A sejtéssel kapcsolatban a legfontosabb előrehaladást tetszőleges $G$ gráf esetén a $\chi_e(G)$-re vonatkozó konstans korlátok bevezetése és javítása jelenti. A sejtést először felvető cikkben még csak azt bizonyították, hogy véges sok valós élsúly elegendő, később viszont egész számokra vonatkozó korlátokat is adtak. A jelenleg ismert legjobb eredmény \citeauthor{Kalkowski2010} nevéhez fűződik, akik a $\chi_e(G) \leq 6$ \cite{Kalkowski2009}, kicsivel később pedig a $\chi_e(G) \leq 5$ \cite{Kalkowski2010} korlátot adták a problémára. A két bizonyítás merőben más eszközöket használ, amelyek önmagukban is említésre érdemesek, ezért a következőkben mindkettőre kitérünk.

\section{Csúcs-színező 6-élsúlyozás}
Először vizsgáljuk a gyengébb korlátot. Az erre vonatkozó tétel bizonyítása előtt tekintsük a következő lemmát, mely az \cite{Kalkowski2009} cikk első szerzőjének egy korábbi eredménye:

\begin{lem}
Minden összefüggő, rendes $G$ gráfra létezik olyan $f:E(G) \rightarrow \lbrace 1, 2, 3 \rbrace$ élsúlyozás és $f':V(G) \rightarrow \lbrace 0, 1 \rbrace$ csúcs-súlyozás, melyre a csúcsok $w(v) = f'(v) + \sum\limits_{w \in N(v)} f(vw)$ értéke egy helyes színezés.
\end{lem}

Ennek segítségével egy $\chi_e(G) \leq 10$ korlát adható az élsúlyok megháromszorozásával, majd bizonyos élek $1$-gyel történő módosításával. Jelen esetben is egy hasonló eljárást követünk majd, amelyhez szükségünk lesz a lemma egy általánosabb alakjára. Előtte azonban érdemes megjegyezni egy egyszerű következményt. A sejtés vizsgálatánál érdekes kérdés lehet, hogy mit tudunk mondani a rossz élek részgráfjáról, vagyis azon élekről, melyek végpontjai azonos értékűek. A fenti lemma erre is ad egyfajta választ, ugyanis az általa biztosított teljes súlyozásban minden csúcsra $0$-t írva olyan élsúlyozást kapunk, ahol a rossz élek egy páros gráfot alkotnak. Ez a megfigyelés segíthet abban, hogy közelebb jussunk a sejtés bizonyításához vagy cáfolatához. Visszatérve a tételünkhöz, a lemma általánosítása a következőképpen hangzik:

\begin{lem}
Legyen $\alpha \in \mathbb{R}$ és $\beta \in \mathbb{R} \smallsetminus \lbrace 0 \rbrace$. Ekkor minden összefüggő, rendes $G$ gráfra, és tetszőleges $T$ feszítőfájára létezik olyan $f:E(G) \rightarrow \lbrace \alpha - \beta, \alpha, \alpha + \beta \rbrace$ élsúlyozás és $f':V(G) \rightarrow \lbrace 0, \beta \rbrace$ csúcs-súlyozás, melyre a csúcsok $w(v) = f'(v) + \sum\limits_{w \in N(v)} f(vw)$ értéke egy helyes színezés. Továbbá $f$ megválasztható úgy, hogy $f(e) = \alpha$ minden $e \in E(T)$-re.
\end{lem}

\begin{proof}
Legyen $v_1, v_2, \ldots, v_n$ a csúcsoknak egy olyan sorrendje, melyre minden $k \geq 2$-re $v_k$-ból pontosan egy $T$-beli él vezet $\lbrace v_1, v_2, \ldots, v_{k - 1} \rbrace$-be. Kezdetben minden élhez az $\alpha$ súlyt rendeljük, amelyet legfeljebb egyszer módosítunk, hogy sorban minden $v_k$ csúcs értékét véglegesítsük.

Legyen $w(v_1) = \alpha d(v_1)$, és tegyük fel, hogy valamely $k \geq 2$-re már meghatároztuk az $f$ élsúlyokat az $E(G\lbrack \lbrace v_1, v_2, \ldots, v_{k - 1} \rbrace \rbrack) \smallsetminus E(T)$ halmazon és az $f'$ csúcs-súlyokat $\lbrace v_1, v_2, \ldots, v_{k - 1} \rbrace$-en úgy, hogy az első $k - 1$ csúcs $w(v_i)$ értéke már végleges.

A $v_k$ csúcs esetén minden $E(v_k, \lbrace v_1, v_2, \ldots, v_{k - 1} \rbrace) \smallsetminus E(T)$-beli él súlyát módosíthatjuk $\beta$-val. Amennyiben $v_k v_i \in E(G) \smallsetminus E(T)$ és $f'(v_i) = 0$, akkor választhatunk $(f(v_k v_i) = \alpha, f'(v_i) = 0)$ és $(f(v_k v_i) = \alpha - \beta, f'(v_i) = \beta)$ között anélkül, hogy megváltoztatnánk $w(v_i)$-t. Hasonlóan, ha $v_k v_i \in E(G) \smallsetminus E(T)$ és $f'(v_i) = \beta$, akkor választhatunk $(f(v_k v_i) = \alpha, f'(v_i) = \beta)$ és $(f(v_k v_i) = \alpha + \beta, f'(v_i) = 0)$ között anélkül, hogy megváltoztatnánk $w(v_i)$-t. Végezetül megválaszthatjuk $f'(v_k)$ értékét is. Ez összesen $|E(v_k, \lbrace v_1, v_2, \ldots, v_{k - 1} \rbrace) \smallsetminus E(T)| + 2 = |E(v_k, \lbrace v_1, v_2, \ldots, v_{k - 1} \rbrace)| + 1$ különböző lehetőség $w(v_k)$ értékének, melyek közül kiválaszthatjuk azt, amely minden $N(v_k) \cap \lbrace v_1, v_2, \ldots, v_{k - 1} \rbrace$-beli csúcs értékétől különbözik.

Ezt az eljárást folytatva megkaphatjuk a kívánt súlyozást.
\end{proof}

Ezen lemma birtokában most már készen állunk a tétel bizonyítására.

\begin{tét}[\citeauthor{Kalkowski2009} \cite{Kalkowski2009}]
Minden $G$ rendes gráfra $\chi_e(G) \leq 6$.
\end{tét}

\begin{proof}
Feltehető, hogy $G$ összefüggő, különben a komponenseket külön-külön vizsgálhatjuk. Induljunk ki egy tetszőleges $T$ feszítőfából, és vegyünk egy $(f, f', w)$ súlyozást a lemma alapján, $\alpha = 4$ és $\beta = -2$ paraméterekkel. Ekkor minden csúcs és él súlya páros. A bizonyítás hátralévő részében módosítani fogjuk $f$-et és $f'$-t, de $w(v)$ változatlan marad minden $v \in V(G)$ csúcsra.

Legyen $H = G\lbrack \lbrace v \in v(G)\ |\ f'(v) = -2 \rbrace \rbrack$, és ebben $H_1$ egy maximális feszítő részgráf, melyben a legnagyobb fokszám legfeljebb $2$. Adjunk hozzá $-1$-et $f(e)$-hez a $H_1$ minden $e$ élére, és módosítsuk $V(H_1)$ minden $v$ csúcsán az $f'(v)$ értéket ennek megfelelően, hogy $w(v)$ változatlan maradjon. Így minden $v \in V(G)$ csúcsra $f'(V) \in \lbrace 0, -1, -2 \rbrace$, minden $e \in E(G)$ élre $f(e) \in \lbrace 1, 2, \ldots, 6 \rbrace$, továbbá minden $e \in E(T)$ élre $f(e) \in \lbrace 3, 4 \rbrace$.

Legyen $i \in \lbrace 0, 1, 2 \rbrace$ esetén $S_i = \lbrace v \in v(G)\ |\ f'(v) = -i \rbrace$ és $s_i = |S_i|$. Figyeljük meg, hogy minden $v \in S_0 \cup S_2$ csúcs $w(v) - f'(v)$ súlya páros, az $S_1$-beli csúcsoké pedig páratlan. $H_1$ maximalitása miatt minden $uv$ élre, ahol $u, v \in S_1 \cup S_2$, teljesül, hogy $u, v \in S_1$ és $uv \in E(H_1)$, hiszen ha nem így lenne, akkor az előző lépésben a $H_1$ részgráfot tudtuk volna még bővíteni. Részletesebben, ezen élek végpontjaira $w(u) - f'(u) \neq w(v) - f'(v)$. Az ilyen élek halmazát jelölje $E^*$.

Ha $s_2 = 0$, akkor készen vagyunk, hiszen $f$ jó színezést ad. Amennyiben $s_2 = 1$ és $s_1 = 0$, legyen $u \in S_2$. Figyeljük meg, hogy minden $u$-ra illeszkedő $e$ él súlya $f(e) \in \lbrace 2, 4, 6 \rbrace$. Ha $u$-nak van egy olyan $v$ szomszédja, melyre $w(u) + 2 \neq w(v)$, akkor az $uv$ és súlyát $1$-gyel csökkentve szintén helyes színezéshez jutunk. (Figyeljük meg, hogy csak $u$ és $v$ súlya páratlan.) Ha $u$ minden $v \in N(u)$ szomszédjára $w(u) + 2 = w(v)$ és $|N(u)| \geq 2$, akkor két különböző, $u$-ra illeszkedő élen is csökkentsük a súlyt $1$-gyel. Ez ismét a kívánt súlyozáshoz vezet. Végül, ha az $u$ csúcs egyetlen $v$ szomszédjára $w(u) + 2 = w(v)$, akkor vegyünk egy $x \in N_T(v) \smallsetminus \lbrace u \rbrace$ csúcsot, csökkentsük $f(uv)$-t $1$-gyel, $f(vx)$-et pedig növeljük $1$-gyel. Így ismét megfelelő súlyozást kapunk.

Ha $s_2 = 1$ és $s_1 \geq 1$, akkor vegyünk egy $T$-beli utat $u \in S_2$ és egy $v \in S_1$ között, majd felváltva csökkentsük és növeljük az élek súlyát $1$-gyel, ügyelve arra, hogy a $v$-re illeszkedő él súlyát csökkentsük. Ezzel a keresett súlyozáshoz jutunk.

Ha $s_2 \geq 2$, akkor indukcióval beláthatjuk, hogy tudunk találni $\lceil \frac{s_2}{2} \rceil$ olyan $T$-beli utat, melyek végpontjai pontosan az $S_2$-beli csúcsok, és amelyek $T$ minden élét legfeljebb kétszer használják. Ilyen utakat $2 \leq s_2 \leq 3$ esetén könnyen találhatunk. Amennyiben $s_2 \geq 4$, úgy keressünk egy olyan $e \in E(T)$ élt, melyre $T-e$ mindkét komponense legalább $2$ $S_2$-beli csúcsot tartalmaz, és legalább az egyikben páros számú ilyen csúcs van. A két komponensre indukciót alkalmazva megtalálhatjuk a keresett utakat.

Felváltva csökkentsük és növeljük ezen utak mentén az élek súlyait úgy, hogy csak a végpontok súlya változzon, és módosítsuk ennek megfelelően az $f'$ értékeket ezeken a csúcsokon. Ha egy $u \in S_2$ csúcs két útnak is végpontja (például, ha $s_2$ páratlan), akkor ügyeljünk arra, hogy az $u$-ra illeszkedő mindkét élen csökkentsük a súlyt, hogy $f'(u) = 0$ adódjon. Figyeljük meg, hogy csak $E(T)$-beli éleket használunk, így nem kapunk $1$-nél kisebb vagy $6$-nál nagyobb élsúlyokat. Ezek után minden csúcsra, amely korábban $S_2$-ben volt, $f'(v) \in \lbrace -3, -1, 0 \rbrace$. Könnyen látható, hogy így az $f$ súlyozást tekintve minden $v$ csúcs értéke $w(v)$, amennyiben $w(v)$ páros. A páratlan értékű csúcsok között futó élek mind $E^*$-ban vannak, tehát a végpontjaik $w$ súlya különböző, ahogyan azt korábban már láttuk. Így $f$ egy csúcs-színező $6$-élsúlyozás.
\end{proof}

\section{Csúcs-színező 5-élsúlyozás}
\begin{tét}[\citeauthor{Kalkowski2010} \cite{Kalkowski2010}]
Minden $G$ rendes gráfra $\chi_e(G) \leq 5$.
\end{tét}

\begin{proof}
Feltehető, hogy $G$ összefüggő, különben komponensenként érvelhetünk. Feltehető még továbbá az is, hogy $|V| \geq 3$, és létezik olyan $v$ csúcs, melyre $d(v) \geq 2$. Legyen $v_1, v_2, \ldots, v_n$ a csúcsoknak egy olyan sorrendje, melyre $d(v_n) \geq 2$, és minden $1 \leq i \leq n-1$-re $v_i$-nek van szomszédja $\lbrace v_{i+1}, v_{i+2}, \ldots, v_n \rbrace$-ben.

Kezdetben minden $e$ élhez az $f(e) = 3$ élsúlyt rendeljük, majd legfeljebb kétszer módosítjuk, miközben sorban végighaladunk a csúcsokon. Minden $i < n$-re a $v_i$ csúcshoz hozzárendelünk két színt, $W_(v_i) = \lbrace w(v_i), w(v_i) +2 \rbrace$, ahol $w(v_i) \in \lbrace 0, 1 \rbrace$ mod $4$, oly módon, hogy minden $v_j v_i \in E$ élre, ahol $1 \leq j < i$, $W(v_j) \cap W(v_i) = \emptyset$, és biztosítani fogjuk, hogy $f(v_i) = \sum\limits_{u \in N(v_i)} f(uv_i) \in W(v_i)$. Végül beállítjuk a $v_n$-re illeszkedő élek súlyát úgy, hogy $f(v_n)$ különbözzön $f(v_i)$-től minden $v_i \in N(v_n)$-re.

Ezt szem előtt tartva legyen $f(v_1) = 3d(v_1)$, és válasszuk meg a $W(v_1)$ halmazt úgy, hogy $f(v_1) \in W(v_1)$, valamint $w(v_1) \in \lbrace 0, 1 \rbrace$ mod $4$ teljesüljön. Legyen $2 \leq k \leq n - 1$, és tegyük fel, hogy már minden $i < k$-ra meghatároztuk $W(v_i)$-t, valamint
\begin{itemize}
\item $f(v_i) \in W(v_i)$, ahol $i < k$
\item $f(v_k v_j) = 3$ minden élre, ahol $j > k$
\item ha $f(v_i v_k) \neq 3$ valamely élre $i < k$ esetén, akkor vagy $f(v_i v_k) = 2$ és $f(v_i) = w(v_i)$, vagy $f(v_i v_k) = 4$ és $f(v_i) = w(v_i) + 2$.
\end{itemize}

Ha $v_i v_k \in E$ valamely $i < k$-ra, akkor $f(v_i v_k)$-t $2$-vel növelhetjük vagy csökkenthetjük úgy, hogy $f(v_i) \in W(v_i)$ maradjon. Amennyiben $v_k$-nak $d$ ilyen szomszédja van, úgy ez $d + 1$ lehetséges értéket jelent $f(v_k)$ számára, melyek mind azonos paritásúak. Ezen felül megengedjük még, hogy az $f(v_k v_j)$ súlyt $1$-gyel módosítsuk, ahol $j > k$ a legkisebb index, melyre $v_k v_j \in E$. Ezáltal $f(v_k)$ egy $\lbrack a, a + 2d + 2 \rbrack$ intervallum minden értékét felveheti. Úgy szeretnénk módosítani a súlyokat és meghatározni $w(v_k)$-t, hogy
\begin{enumerate}
\item $f(v_i) \in W(v_i)$, ahol $1 \leq i \leq k$
\item $v_i v_k \in E$ esetén $w(v_i) \neq w(v_k)$, ahol $i < k$
\item vagy $f(v_k) = w(v_k)$ és $f(v_k v_j) \in \lbrace 2, 3 \rbrace$ vagy $f(v_k) = w(v_k) + 2$ és $f(v_k v_j) \in \lbrace 3, 4 \rbrace$
\end{enumerate}
teljesüljön. A második feltétel legfeljebb $2d$ értéket zárhat ki az $\lbrack a, a + 2d + 2 \rbrack$ intervallumból, míg a harmadik feltétel csak az $a$ és $a + 2d + 2$ értékeket, hiszen minden más $f(v_k)$ értékre $f(v_k v_j) \neq 3$ esetén lehetőségünk van választani $f(v_k v_j) = 2$ és $f(v_k v_j) = 4$ között. Így legalább egy érték szabadon marad $f(v_k)$ számára.

Ilyen módon lépésről lépésre, konfliktus nélkül meghatározhatjuk a $W(v_k)$ halmazokat minden $k \leq n - 1$-re. Vegyük észre, hogy amikor az $f(v_k)$ érték először változik meg egy $v_k v_i$, $i > k$ él módosítása miatt, akkor $i = j$, vagyis nem okoznak problémát a $2$ vagy $4$ súlyú élek.

Utolsó lépésként találnunk kell egy szabad értéket $v_n$-nek. Ez alkalommal nem áll rendelkezésünkre egy $v_n v_j$ segédél, de nem is kell későbbi csúcsok miatt aggódnunk. Az előzőekhez hasonlóan, ha $v_i v_n \in E$ valamely $i < n$-re, akkor $f(v_i v_n)$-t $2$-vel növelhetjük vagy csökkenthetjük úgy, hogy $f(v_i) \in W(v_i)$ maradjon. Ezek a módosítások összesen $d(v_n) +1 \geq 3$, azonos paritású lehetőséget jelentenek $f(v_n)$ értékének. Így, ha a legkisebb ilyen lehetséges $a$ értékre $a \in \lbrace 2, 3 \rbrace$ mod $4$, akkor minden $v_n$-re illeszkedő élen a kisebb értéket választva a csúcsok egy helyes színezését kapjuk. Ha $a \in \lbrace 0, 1 \rbrace$ mod $4$, és létezik olyan $v_i \in N(v_n)$ csúcs, melyre $w(v_i) \neq a$, akkor a $v_i v_n$ élen a nagyobb, minden más élen pedig a kisebb súlyt választva $f(v_n) = a + 2$, ami szintén helyes színezéshez vezet. Végezetül, amennyiben $a \in \lbrace 0, 1 \rbrace$ mod $4$ és $w(v_i) = a$ minden $v_i \in N(v_n)$-re, akkor legalább két élen a nagyobb súlyt választva kapunk helyes színezést. Ezzel a tétel állítását beláttuk.
\end{proof}

\chapter{Speciális esetek}
Habár az 1, 2, 3 - sejtést még nem sikerült bizonyítani, bizonyos gráfosztályokra már belátták, hogy létezik csúcs-színező $3$-élsúlyozásuk. A továbbiakban ezeket fogjuk megvizsgálni.

\section{Színezés $χ(G)$ élsúllyal}
Az első ilyen típusú eredmény a sejtést először felvető cikkből \cite{Karonski2004} származik. Ez azt mondja ki, hogy egy $k$-színezhető gráf élei megsúlyozhatóak egy $k$-adrendű Abel-csoport elemeivel csúcs-színező módon, amennyiben $k$ páratlan. Ebből rögtön következik, hogy minden $3$-színezhető gráfra igaz a sejtés. Az alábbi két tétel ezen eredmény módosítása, melyet \citeauthor{Lu2011} \cite{Lu2011} cikkében olvashatunk.

\begin{tét}
Legyen $G$ egy összefüggő nem-páros gráf és $\Gamma = \lbrace g_1, g_2, \ldots, g_k \rbrace$ egy véges Abel-csoport, ahol $k = |\Gamma|$. Legyen továbbá $s$ egy $k$-színezése a $G$ csúcsainak az $\lbrace U_1, U_2, \ldots, U_k \rbrace$  színosztályokkal, ahol $|U_i| = n_i$, $1 \leq i \leq k$. Ha létezik olyan $h \in \Gamma$, melyre $n_1 g_1 + \cdots + n_k g_k = 2h$, akkor létezik olyan élsúlyozás $\Gamma$ elemeivel, melyre az indukált csúcs-színezés $s$.
\end{tét}

\begin{proof}
Legyen $s$ egy $k$-színezés a $g_1, g_2, \ldots, g_k$ színekkel és az $\lbrace U_1, U_2, \ldots, U_k \rbrace$ színosztályokkal, melyre $n_1 g_1 + \cdots + n_k g_k = 2h$.

Tegyünk egy élre $h$ súlyt, a többire pedig $0$-t, így a csúcsszínek összege $2h$. A következőkben ezt az élsúlyozást fogjuk módosítani úgy, hogy közben ez az összeg ne változzon, amíg minden $U_i$-beli csúcs színe $g_i$ nem lesz, $1 \leq i \leq k$-ra. Tegyük fel, hogy létezik egy $u \in U_i$ csúcs, amelynek a $g \neq g_i$ színe nem megfelelő. Mivel $n_1 g_1 + \cdots + n_k g_k = 2h$, ezért szükségképpen létezik egy $u$-tól különböző $v$ csúcs, amelynek szintén rossz a színe. Válasszunk egy páros hosszú sétát $u$-ból $v$-be. Ez mindig megtehető, mivel $G$ nem-páros és $k \geq 3$. Adjuk hozzá a séta éleihez felváltva a $g_i - g$ illetve a $g - g_i$ értéket. Ez az eljárás megtartja a csúcsszínek összegét, valamint minden csúcs színét $u$ és $v$ kivételével, továbbá eggyel növeli a megfelelő színű csúcsok számát. Ennek ismételt alkalmazásával megkaphatjuk a kívánt súlyozást.
\end{proof}

Érdemes megjegyezni, hogy a fenti tételben $s$ tetszőleges színezés lehet, nem csak egy helyes színezése a csúcsoknak.

\begin{tét}
Legyen $G$ egy rendes, összefüggő páros gráf és $Z_2 = \lbrace 0, 1 \rbrace$. Legyen továbbá $s$ egy $2$-színezése a $G$ csúcsainak az $\lbrace U_0, U_1 \rbrace$ színosztályokkal, ahol $|U_i| = n_i$, $i = 0, 1$. Ha $n_1$ páros, akkor létezik olyan élsúlyozás $Z_2$ elemeivel, melyre az indukált csúcs-színezés $s$.
\end{tét}

\begin{proof}
Kövessük az előző bizonyítás gondolatmenetét, és tegyünk egy élre $h = 1$ súlyt. Ha létezik egy $u \in U_i$ csúcs, amelynek nem megfelelő a színe, akkor $n_1$ párossága miatt szükségképpen létezik egy $u$-tól különböző $v$ csúcs, amelynek szintén rossz a színe. Mivel $G$ összefüggő, ezért létezik út $u$-ból $v$-be. Adjunk hozzá az út minden éléhez $1$-et. Ez az eljárás megtartja a csúcsszínek összegét, valamint minden csúcs színét $u$ és $v$ kivételével, továbbá eggyel növeli a megfelelő színű csúcsok számát. Ennek ismételt alkalmazásával megkaphatjuk a kívánt súlyozást.
\end{proof}

Ezzel beláttuk, hogy $3$-színezhető gráfnak van csúcs-színező $3$-élsúlyozása. Felmerül a kérdés, hogy hasonló állítás igaz-e páros gráfokra. A válasz sajnos nem, ugyanis könnyen ellenőrizhető, hogy például a $C_6$ vagy $C_{10}$ gráfoknak nincs ilyen súlyozásuk. A második tétel alapján viszont az alábbi állítást fogalmazhatjuk meg:

\begin{áll}
Legyen $G = (U, V; E)$ egy rendes, összefüggő páros gráf. Ha $|A|$ vagy $|B|$ páros, akkor $G$-nek létezik csúcs-színező $2$-élsúlyozása.
\end{áll}

\chapter{Élsúlyozások irányított gráfokon}
Az eddigiekben irányítatlan gráfok élsúlyozásait vizsgáltuk, de joggal merülhet fel a kérdés, hogy vajon mit tudunk mondani az irányított esetről. Itt két lehetőségünk van egy csúcs színének meghatározására. 

Az első esetben a kimenő élek összsúlyából kivonjuk bemenő élek összsúlyát. Erről a változatról \citeauthor{Bartnicki2009} \cite{Bartnicki2009} bebizonyították, hogy a súlyokat tetszőleges kételemű listákról választva is létezik csúcs-színező élsúlyozás.

A második esetben egy csúcs színét csak a kimenő éleken vett súlyok összege, azaz a csúcs súlyozott kifoka határozza meg. A továbbiakban ezzel az esettel fogunk foglalkozni.

\section{Csúcs-színező 3-élsúlyozás}
Könnyen látható, hogy léteznek olyan digráfok, például a $3$ hosszú irányított kör, amelyek helyes színezéséhez nem elegendőek az $\lbrace 1, 2 \rbrace$ élsúlyok. A következő tétel azt mondja ki, hogy minden irányított gráfnak van csúcs-színező 3-élsúlyozása. Ez abból következik, hogy minden digráfnak van egy alkalmas csúcsa, amely a szomszédai számához képest sok lehetséges súlyozott kifok-értéket vehet fel. Egy ilyen csúcs létezése teljes indukció használatát teszi lehetővé, továbbá a bizonyítás polinomiális idejű algoritmust is eredményez egy csúcs-színező 3-élsúlyozás megtalálására.

\begin{tét}[\citeauthor{Baudon2014} \cite{Baudon2014}]
Minden $D$ digráfra $\chi_e(D) \leq 3$.
\end{tét}

\begin{proof}
A bizonyítás $D$ élszáma szerinti indukcióval történik. Az állítás nyilvánvaló $0$ vagy $1$ élű digráf esetén. Tegyük fel, hogy legfeljebb $m - 1$ élre már igazoltuk a tételt, és legyen $D = (V, A)$ egy $m \geq 2$ élű digráf.

Figyeljük meg, hogy $D$-nek létezik egy olyan $v$ csúcsa, melyre $\delta(v) > 0$ és $\delta(v) > \varrho(v)$, hiszen különben $\sum\limits_{v \in V} \varrho(v) \neq \sum\limits_{v \in V} \delta(v)$ lenne. Első lépésként töröljünk minden $v$-ből kilépő élet. Ekkor az indukciós feltevés miatt a fennmaradó digráfnak létezik egy $w$ csúcs-színező 3-élsúlyozása. Tegyük vissza a kitörölt éleket, és terjesszük ki ezekre $w$-t oly módon, hogy $v$ súlyozott kifoka különbözzön mind a $\delta(v) + \varrho(v)$ szomszédjáétól. Ez megtehető, hiszen $2\delta(v) + 1$ érték közül választhatunk, nevezetesen a $\lbrace \delta(v), \delta(v) + 1, \ldots, 3\delta(v) \rbrace$ halmazból, míg a tiltott értékek száma $v$ választása miatt legfeljebb $\varrho(v) + \delta(v) < 2\delta(v) + 1$. Mivel ezen élsúlyok kizárólag a $v$ csúcs súlyozott kifokát befolyásolják, ezért az így kapott kiterjesztett súlyozás a $D$ digráf egy csúcs-színező 3-élsúlyozása.
\end{proof}

Ebben a bizonyításban azt használtuk ki, hogy egy $d$-edfokú csúcs lehetséges súlyozott kifokainak száma kellően nagy, pontosabban legalább $2d + 1$, amennyiben az éleket az $\lbrace 1, 2, 3 \rbrace$ számokkal súlyozzuk. Most megmutatjuk, hogy ez a tulajdonság tetszőleges $\lbrace a, b, c \rbrace$ súlyok esetén fennáll, és így egy erősebb tétel is igaz.

\begin{lem}
Legyen egy $D$ digráfnak $v$ egy legalább $d$-edfokú csúcsa, valamint $a, b$ és $c$ három valós szám. Ekkor $v$ súlyozott kifoka legalább $2d + 1$ különböző értéket vehet fel $D$ egy tetszőleges élsúlyozásában, amelyben a $v$-ből kimenő élek súlyai az $\lbrace a, b, c \rbrace$ halmazból kerülnek ki.
\end{lem}

\begin{proof}
A bizonyítás $d$ szerinti indukcióval történik. Amennyiben $d = 1$, úgy a $v$-ből kilépő él súlya $a, b$ vagy $c$ lehet. Mivel ezek különbözőek, ezért $v$ súlyozott kifokának is $3$ különböző értéke lehet.

Tegyük fel, hogy $d \leq i - 1$ esetén már igazoltuk az állítást, és legyen $d = i$. Jelölje $D'$ azt a digráfot, amelyet egy $v$-ből kilépő $vu$ él elhagyásával kapunk $D$-ből. Ekkor az indukciós feltevés szerint $v$ súlyozott kifoka legalább $2(d - 1) + 1$ lehetséges értéket vehet fel $D$ egy tetszőleges élsúlyozásában, amely a $v$-ből kilépő éleket az $\lbrace a, b, c \rbrace$ számokkal súlyozza. Legyen $F'$ ezen lehetséges értékek halmaza, valamint $k$ és $n$ rendre a legkisebb, illetve legnagyobb eleme $F'$-nek, továbbá $w_K$ és $w_N$ két élsúlyozása $D'$-nek, melyekre a $v$ csúcs súlyozott kifoka rendre $K$, illetve $N$.

Tegyük fel, hogy $a < b < c$. Amennyiben az állítás igaz az $\lbrace a, b, c \rbrace$ számokra, úgy igaz a $\lbrace -a, -b, -c \rbrace$ számokra is, ezért két eset lehetséges:

\begin{enumerate}
\item $0 \leq a < b < c$
\item $a < 0 \leq b < c$
\end{enumerate}

Az első esetben terjesszük ki a $D'$ minden élsúlyozását a $D$ digráfra úgy, hogy a $vu$ élre $a$ súlyt írunk. Ekkor azt kapjuk, hogy az $F = \lbrace x + a: x \in F' \rbrace$ halmaz a $v$ csúcs lehetséges súlyozott kifokainak $2(d - 1) + 1$ elemű halmaza. A fennmaradó két lehetőséget úgy kapjuk, hogy a $w_N$ súlyozást $b$ vagy $c$ értékkel kiterjesztjük a $vu$ élre. Ekkor ugyanis $N + b$ és $N + c$ két újabb lehetséges érték, hiszen $a < b < c$ miatt ezek nincsenek $F$-ben. Tehát létezik legalább $2d + 1$ választás $v$ súlyozott kifokára.

A második esetben terjesszük ki a $D'$ minden élsúlyozását a $D$ digráfra úgy, hogy a $vu$ élre $b$ súlyt írunk. Ekkor azt kapjuk, hogy az $F = \lbrace x + b: x \in F' \rbrace$ halmaz a $v$ csúcs lehetséges súlyozott kifokainak $2(d - 1) + 1$ elemű halmaza. A fennmaradó két lehetőséget úgy kapjuk, hogy a $w_N$ és $w_K$ súlyozásokat rendre $a$, illetve $c$ értékkel kiterjesztjük a $vu$ élre. Ezekből azt kapjuk, hogy $K + a$ és $N + c$ két újabb lehetséges érték, amely nem szerepel $F$-ben a súlyokra vonatkozó feltevés miatt. Ezzel az állítást beláttuk.
\end{proof}

A lemma következményeként azt kapjuk, hogy az előző tétel bizonyításában nem számít, hogy egy csúcsnál mely három súlyt írhatjuk a kimenő élekre. Ez a megfigyelés az alábbi listaszínezési tételhez vezet.

\begin{tét}
Legyen adott egy $D$ digráf minden $v$ csúcsára egy három valós számból álló $L(v)$ lista. Ekkor $D$-nek van olyan csúcs-színező élsúlyozása, amelyben minden $v$ csúcsra a kimenő élek súlyai $L(v)$-ből kerülnek ki.
\end{tét}

A probléma egyfajta kiterjesztéseként megkérdezhetjük, hogy melyik az a legkisebb $k \in \lbrace 1, 2, 3 \rbrace$, melyre minden irányított gráfnak létezik olyan irányítása, amelynek van csúcs-színező $k$-élsúlyozása. Tudjuk, hogy egy digráfnak pontosan akkor létezik csúcs-színező $1$-élsúlyozása, ha bármely két szomszédos pont kifoka különböző. A következő lemma azt mondja ki, hogy minden gráfnak létezik olyan irányítása, amelyre ez teljesül.

\begin{lem}
Minden $G$ gráfnak létezik olyan irányítása, amelyben bármely két szomszédos csúcs kifoka különböző.
\end{lem}

\begin{proof}
A bizonyítás $G$ pontszáma szerinti indukcióval történik. Az állítás $n \leq 2$ esetén nyilvánvaló. Tegyük fel, hogy legfeljebb $i - 1$ csúcsra már igaz a lemma, és legyen $G$ egy $n = i$ pontú gráf. Jelölje $v$ a legnagyobb fokszámú csúcsot $G$-ben. Az indukciós feltevés szerint $G' = G - v$ megirányítható úgy, hogy bármely két szomszédos csúcs kifoka különböző legyen. Jelölje az így kapott digráfot $D'$. Figyeljük meg, hogy $v$ választása miatt $D'$-ben minden $v$-vel szomszédos csúcs kifoka legfeljebb $d(v) - 1$. Legyen $D$ a $G$ gráf azon irányítása, amelyet a $D'$ irányításból kapunk oly módon, hogy a $v$-re illeszkedő éleket mind kifelé irányítjuk. Mivel így $v$ kifoka a $D$ digráfban $d(v)$, és a szomszédos csúcsok kifoka nem változott, ezért a kapott irányítás továbbra is teljesíti a lemma feltételét.
\end{proof}

Ezen lemma ismeretében és a korábbi megfigyelésünk alapján az alábbi tételt fogalmazhatjuk meg:

\begin{tét}
Minden irányítatlan gráfnak létezik olyan irányítása, amelynek van csúcs-színező $1$-élsúlyozása.
\end{tét}

\nocite{*}
\printbibliography

\end{document}